%%%%%%%%%%%%%%%%%%%%%%%%%%%%%%%%%%%%%%%%%%%%%%%%%%%%%%%%%%%%%%%%%%%%%%%%%%%%
%%
%A  cryst.tex                   GAP documentation                Bettina Eick
%A                                                            & Franz Gaehler
%A                                                            & Werner Nickel
%%
%Y  Copyright 1990-1997,  Lehrstuhl D fuer Mathematik,  RWTH Aachen,  Germany
%%

%%%%%%%%%%%%%%%%%%%%%%%%%%%%%%%%%%%%%%%%%%%%%%%%%%%%%%%%%%%%%%%%%%%%%%%%%%%
\Chapter{Affine crystallographic groups}

An affine crystallographic group <G> is a subgroup of the group of all
Euclidean motions of $d$-dimensional space, with the property that its
subgroup <T> of all pure translations is a discrete normal subgroup of
finite index. If the rank of the translation subgroup <T> is $d$, <G>
is called a space group. The quotient <G>/<T> is called the point group
of <G>.

In this package, affine crystallographic groups are represented as
groups of augmented matrices of dimension $d+1$. This is possible
in two ways. Augmented matrices of the form
\begintt
                     [ M 0 ]
                     [ t 1 ]
\endtt
act from the right on row vectors $(x,1)$. Such a matrix is said to
be an affine matrix acting on the right. Since in {\GAP} all groups
act from the right, this is the preferred representation of an affine
transformation. 

The second representation of affine transformations is by augmented
matrices of the form 
\begintt
                     [ M t ]
                     [ 0 1 ]
\endtt
which act from the left on column vectors $(x,1)$. Such matrices are
said to be affine matrices acting on the left. This is the
representation usually adopted by crystallographers. 

{\Cryst} supports affine crystallographic groups in both
representations. Every affine crystallographic group is 
constructed in one of these two representations. 

Affine crystallographic groups in different representations should
never be mixed, however. It is recommended to adopt one of the two
representations, and then to stick to that decision. In order to
facilitate this, there is a global variable `CrystGroupDefaultAction',
whose value is either `RightAction' or `LeftAction'. The initial value
is `RightAction', but this can be changed with

\> SetCrystGroupDefaultAction( <action> ) F

where <action> must be either `RightAction' or `LeftAction'.
Constructor functions without explicit representation
qualifier then will construct an affine crystallographic group
in the representation specified by `CrystGroupDefaultAction'.


%%%%%%%%%%%%%%%%%%%%%%%%%%%%%%%%%%%%%%%%%%%%%%%%%%%%%%%%%%%%%%%%%%%%%%%%%%%
\Section{Construction}

\atindex{construction!of an AffineCrystGroup}%
{@construction!of an \noexpand`AffineCrystGroup'}

\> AffineCrystGroupOnRight( <gens> ) F
\> AffineCrystGroupOnRight( <genlist> ) F
\> AffineCrystGroupOnRight( <genlist>, <identity> ) F

returns the matrix group generated by <gens> or <genlist>, which must
be affine matrices acting on the right, as affine crystallographic
group acting on the right. An already existing group <S> of affine
matrices acting on the right can be converted to an affine
crystallographic group acting on the right with

\>AsAffineCrystGroupOnRight( <S> ) F

The property

\>IsAffineCrystGroupOnRight( <S> ) P

is `true' exactly for those groups which have been constructed in the 
above two ways.

\>AffineCrystGroupOnLeft( <gens> ) F
\>AffineCrystGroupOnLeft( <genlist> ) F
\>AffineCrystGroupOnLeft( <genlist>, <identity> ) F

returns the matrix group generated by <gens> or <genlist>, which must
be affine matrices acting on the left, as affine crystallographic
group acting on the left. An already existing group <S> of affine
matrices acting on the left can be converted to an affine
crystallographic group acting on the left with

\>AsAffineCrystGroupOnLeft( <S> ) F

The property

\>IsAffineCrystGroupOnLeft( <S> ) P

is `true' exactly for those groups which have been constructed in the 
above two ways.

It is recommended to adopt one representation for affine
crystallographic groups, and then to stick to it. To facilitate
this, routines are provided which assume a default representation.

\>AffineCrystGroup( <gens> ) F
\>AffineCrystGroup( <genlist> ) F
\>AffineCrystGroup( <genlist>, <identity> ) F

calls `AffineCrystGroupOnRight' or `AffineCrystGroupOnLeft' with
the same arguments, depending on the value of `CrystGroupDefaultAction'.

\>AsAffineCrystGroup( <S> ) F

calls `AsAffineCrystGroupOnRight' or `AsAffineCrystGroupOnLeft' with
the same argument, depending on the value of `CrystGroupDefaultAction'.

\>IsAffineCrystGroup( <S> ) F

calls `IsAffineCrystGroupOnRight' or `IsAffineCrystGroupOnLeft' with
the same argument, depending on the value of `CrystGroupDefaultAction'.

\>TransposedMatrixGroup( <S> ) A

returns the transpose of the affine crystallographic group <S>. 
If <S> is acting on the right, its transpose is acting on the left,
and vice versa.


%%%%%%%%%%%%%%%%%%%%%%%%%%%%%%%%%%%%%%%%%%%%%%%%%%%%%%%%%%%%%%%%%%%%%%%%%
\Section{Point group}

\atindex{point group!of an AffineCrystGroup}%
{@point group!of an \noexpand`AffineCrystGroup'}

The point group <P> of an affine crystallographic group <S> is the
quotient <S>/<T>, where <T> is the normal subgroup of all pure 
translations. <P> is isomorphic to the group generated by the
linear parts of all affine matrices contained in <S>. In {\Cryst}
this latter group is identified with the point group of <S>.

\>PointGroup( <S> ) A

returns the point group of <S>.

\>IsPointGroup( <P> ) P

returns `true' if and only if <P> has been constructed as the
point group of an affine crystallographic group <S>.

\>AffineCrystGroupOfPointGroup( <P> ) A

returns the affine crystallographic group <S>, from which <P> has
been constructed.

\>PointHomomorphism( <S> ) A

returns a homomorphism from the affine crystallographic group to its
point group.

\>IsPointHomomorphism( <H> ) P

returns `true' if and only if <H> has been constructed as the 
`PointHomomorphism' of an affine crystallographic group.


%%%%%%%%%%%%%%%%%%%%%%%%%%%%%%%%%%%%%%%%%%%%%%%%%%%%%%%%%%%%%%%%%%%%%%%%%
\Section{Translation lattice}

\atindex{translation lattice!of an AffineCrystGroup}%
{@translation lattice!of an \noexpand`AffineCrystGroup'}

The vectors by which the pure translations in an affine
crystallographic group translate form a discrete lattice, <L>,
called the translation lattice of <S>.

\>TranslationBasis( <S> ) A

returns a basis of the translation lattice of <S>. The basis returned
is unique for the translation lattice.

\>InternalBasis( <S> ) A

returns a basis used internally for many computations. It consists
of the translation basis <B> of <S>, extended by further standard
basis vectors if <B> has not full rank. 

If a generating set <B> of the translation lattice of <S> is known
from somewhere, this knowledge can be added to <S> with
 
\>AddTranslationBasis( <S>, <B> ) F

This function must do further work, so that `SetTranslationBasis'
cannot be used for this purpose.  If doubts arise about the
correctness of the translation basis that has been added by hand, 
one can check the correctness of the stored value with

\>CheckTranslationBasis( <S> ) F

An affine crystallographic group <S> acting on $d$-dimensional Euclidean
space is called a *space group* if its translation lattice has rank $d$.

\>IsSpaceGroup( <S> ) P

tests if the affine crystallographic group <S> is a space group.

Since many computations are done internally in the `InternalBasis' of
<S>, we say that <S> is in standard form if the `InternalBasis' is the
standard basis of Euclidean row space or column space, respectively. 
This means that the translation lattice is generated by the first $k$ 
standard basis vectors, where $k$ is the rank of the translation lattice.

\>IsStandardAffineCrystGroup( <S> ) P

checks if <S> is in standard form.

\>IsStandardSpaceGroup( <S> ) P

checks if <S> is a space group in standard form.

\>StandardAffineCrystGroup( <S> ) F

returns a conjugate of <S> which is in standard form.

If an space group is a semi-direct product of its point group with its
translation subgroup, <S> is said to be symmorphic.

\>IsSymmorphicSpaceGroup( <S> ) P

checks if the space group <S> is symmorphic.


%%%%%%%%%%%%%%%%%%%%%%%%%%%%%%%%%%%%%%%%%%%%%%%%%%%%%%%%%%%%%%%%%%%%%%%%
\Section{Special methods}

\atindex{methods!for an AffineCrystGroup}%
{@methods!for an \noexpand`AffineCrystGroup'}

In the representation by augmented matrices, affine crystallographic 
groups are infinite matrix groups. Their infinity is relatively
trivial in the sense that they have an abelian normal subgroup of 
finite index. Nevertheless, for many operations special methods
have to be installed that avoid to attempt algorithms that never
finish. These methods all make essential use of the exactness of the
sequence of homomorphism `0 -> <T> -> <S> -> <P> -> 1', where
<T> is the translation subgroup of <S>, and <P> its point group.

All operations for general groups that make sense for affine
crystallographic groups should work also in that case. In 
particular, there should be no restrictions for finite 
`AffineCrystGroups'. For infinite groups, some restrictions 
apply, however. For instance, algorithms from the orbit-stabilizer 
family can work only if the orbits generated are finite. Note,
however, that `Normalizer', `Centralizer' and `RepresentativeAction'
in an `AffineCrystGroup' work even if the corresponding orbit is
infinite.

Some methods installed for affine crystallographic groups have
a special behavior.

\indextt{\\\^{}!for an \noexpand`AffineCrystGroup'}
\>`\\^( <S>, <conj> )'{power!for an `AffineCrystGroup'}%
@{power!for an `AffineCrystGroup'}

If <S> is an `AffineCrystGroupOnRight', the group <conj * S * conj^{-1}> 
is returned. <conj> must be an affine matrix acting on the right.
If <S> is an `AffineCrystGroupOnLeft', the group <conj^{-1} * S * conj> 
is returned. <conj> must be an affine matrix acting on the left.

\>IsomorphismFpGroup( <P> )!{for a `PointGroup'} A

returns an isomorphism from the `PointGroup' <P> to an isomorphic
`FpGroup' <F>. If <P> is solvable, <F> is given in a 
power-commutator presentation. 

\>IsomorphismFpGroup( <S> )!{for an `AffineCrystGroup'} A

returns an isomorphism from the `AffineCrystGroup' <S> to an isomorphic
`FpGroup' <F>. If <S> is solvable, <F> is given in a power-commutator 
presentation. The presentation of <F> is an extension of the presentation
of the point group <P> of <S> used in `IsomorphismFpGroup( <P> )'. 

If the package polycyclic is installed, {\Cryst} automatically loads
it, and then provides special methods for `IsomorphismPcpGroup'.

\>IsomorphismPcpGroup( <P> )!{for a `PointGroup'} A

with <P> a solvable `PointGroup', returns an isomorphism from <P> 
to an isomorphic `PcpGroup' <pcp>. For details about `PcpGroups', 
we refer to the documentation of the package polycyclic.  

\>IsomorphismPcpGroup( <S> )!{for an `AffineCrystGroup'} A

with <S> a solvable `AffineCrystGroup' (i.e., one with a solvable
`PointGroup'), returns an isomorphism from <S> to an isomorphic
`PcpGroup' <pcp>. The presentation of <pcp> is an extension of the 
presentation of the point group <P> of <S> used in 
`IsomorphismPcpGroup( <P> )'. 

%%%%%%%%%%%%%%%%%%%%%%%%%%%%%%%%%%%%%%%%%%%%%%%%%%%%%%%%%%%%%%%%%%%%%%%%%
\Section{Maximal subgroups}

\atindex{subgroups!maximal!for an AffineCrystGroup}%
{@subgroups!maximal!for an \noexpand`AffineCrystGroup'}
\atindex{maximal subgroups!for an AffineCrystGroup}%
{@maximal subgroups!for an \noexpand`AffineCrystGroups'}

Since an `AffineCrystGroup' has infinitely many maximal subgroups in general,
in the computation of maximal subgroups it must be further specified which
maximal subgroups are desired. Recall that a maximal subgroup of an
`AffineCrystGroup' is either latticeequal or classequal. A latticeequal 
subgroup has the same translation lattice as the parent, while a 
classequal subgroup has the same point group as the parent. In the 
classequal case a maximal subgroup always has prime-power index, whereas
in the latticeequal case this is so only in dimensions up to 3.

\>MaximalSubgroupClassReps( <S>, <flags> )!{for an `AffineCrystGroup'} O

returns a list of conjugacy class representatives of maximal subgroups
of the `AffineCrystGroup' <S>.

\>ConjugacyClassesMaximalSubgroups( <S>, <flags> )!{for an `AffineCrystGroup'} O

returns a list of conjugacy classes of maximal subgroups
of the `AffineCrystGroup' <S>.

In these two functions, the argument <flags> specifies which maximal
subgroups are computed. <flags> is a record which may have the following
components:

\beginitems

   `flags.primes := [p1 .. pr]' &
   only maximal subgroups of p-power index for the given primes p are 
   computed 

   `flags.latticeequal := true' &
   only latticeequal maximal subgroups are computed

   `flags.classequal := true' &
   only classequal maximal subgroups are computed

\enditems

`flags.latticeequal' and `flags.classequal' must not both be bound
and `true'. `flags.primes' may be omitted only if `flags.latticeequal' 
is bound and `true'.

\beginexample
gap> S := SpaceGroupIT(3,222);
SpaceGroupOnRightIT(3,222,'2')
gap> L := MaximalSubgroupClassReps( S, rec( primes := [3,5] ) );
[ <matrix group with 7 generators>, <matrix group with 8 generators>, 
  <matrix group with 8 generators> ]
gap> List( L, IndexInParent );
[ 3, 27, 125 ]
gap> L := MaximalSubgroupClassReps( S,             
>                  rec( classequal := true, primes := [3,5] ) );
[ <matrix group with 8 generators>, <matrix group with 8 generators> ]
gap> List( L, IndexInParent );                                                 
[ 27, 125 ]
gap> L := MaximalSubgroupClassReps( S,
>                  rec( latticeequal := true, primes := [3,5] ) );
[ <matrix group with 7 generators> ]
gap> List( L, IndexInParent );                                       
[ 3 ]
gap> L := MaximalSubgroupClassReps( S, rec( latticeequal := true ) );
[ <matrix group with 7 generators>, <matrix group with 7 generators>, 
  <matrix group with 7 generators>, <matrix group with 7 generators>, 
  <matrix group with 6 generators> ]
gap> List( L, IndexInParent );                                       
[ 2, 2, 2, 3, 4 ]
\endexample

%%%%%%%%%%%%%%%%%%%%%%%%%%%%%%%%%%%%%%%%%%%%%%%%%%%%%%%%%%%%%%%%%%%%%%%%%
\Section{Space groups with a given point group}

\index{space groups!for given point group}

\>SpaceGroupsByPointGroupOnRight( <P> ) O
\>SpaceGroupsByPointGroupOnRight( <P>, <norm> ) O
\>SpaceGroupsByPointGroupOnRight( <P>, <norm>, <orbsflag> ) O

where <P> is any finite subgroup of $GL(d,\Z)$, returns a list of 
all space groups (acting on the right) with point group <P>, up to 
conjugacy in the full translation group of Euclidean space. All 
these space groups are returned as `AffineCrystGroupOnRight' in 
standard representation.  If a second argument is present, which must 
be a list of elements of the normalizer of <P> in $GL(d,\Z)$, only 
space groups inequivalent under conjugation with these elements are 
returned.  If these normalizer elements, together with <P>, generate 
the full normalizer of <P> in $GL(d,\Z)$, then exactly one 
representative of each space group type is obtained. 
If the third argument <orbsflag>, which must be `false' or `true',
is also present and `true', all space groups up to conjugacy in
the full translation group are returned, but these space groups are 
collected into orbits under the conjugation action with elements from
<norm>.

\beginexample
gap> P := Group([ [ [ -1, 0 ], [ 0, -1 ] ], [ [ -1, 0 ], [ 0, 1 ] ] ]);
Group([ [ [ -1, 0 ], [ 0, -1 ] ], [ [ -1, 0 ], [ 0, 1 ] ] ])
gap> norm := GeneratorsOfGroup( NormalizerInGLnZ( P ) );
[ [ [ -1, 0 ], [ 0, -1 ] ], [ [ -1, 0 ], [ 0, 1 ] ], [ [ -1, 0 ], [ 0, -1 ] ],
  [ [ 1, 0 ], [ 0, -1 ] ], [ [ 0, 1 ], [ 1, 0 ] ] ]
gap> SpaceGroupsByPointGroupOnRight( P );
[ <matrix group with 4 generators>, <matrix group with 4 generators>, 
  <matrix group with 4 generators>, <matrix group with 4 generators> ]
gap> SpaceGroupsByPointGroupOnRight( P, norm );
[ <matrix group with 4 generators>, <matrix group with 4 generators>, 
  <matrix group with 4 generators> ]
gap> SpaceGroupsByPointGroupOnRight( P, norm, true );
[ [ <matrix group with 4 generators> ], 
  [ <matrix group with 4 generators>, <matrix group with 4 generators> ], 
  [ <matrix group with 4 generators> ] ]
\endexample

\>SpaceGroupTypesByPointGroupOnRight( <P> ) O
\>SpaceGroupTypesByPointGroupOnRight( <P>, <orbsflag> ) O

returns a list of space group type representatives (acting on the right) 
of the point group <P>. As in the case of `SpaceGroupsByPointGroupOnRight',
if the boolean argument <orbsflag> is present and `true', not only space
group type representatives, but all space groups up to conjugacy in
the full translation group are returned. These are then collected 
into lists of space groups of the same space group type.

\beginexample
gap> SpaceGroupTypesByPointGroupOnRight( P );
[ <matrix group with 4 generators>, <matrix group with 4 generators>, 
  <matrix group with 4 generators> ]
gap> SpaceGroupTypesByPointGroupOnRight( P, true );
[ [ <matrix group with 4 generators> ], 
  [ <matrix group with 4 generators>, <matrix group with 4 generators> ], 
  [ <matrix group with 4 generators> ] ]
\endexample

\>SpaceGroupsByPointGroupOnLeft( <P> ) O
\>SpaceGroupsByPointGroupOnLeft( <P>, <norm> ) O
\>SpaceGroupsByPointGroupOnLeft( <P>, <norm>, <orbsflag> ) O

works the same way as `SpaceGroupsByPointGroupOnRight', except that
the space groups acting from the left are returned.

\>SpaceGroupTypesByPointGroupOnLeft( <P> ) O
\>SpaceGroupTypesByPointGroupOnLeft( <P>, <orbsflag> ) O

works the same way as `SpaceGroupTypesByPointGroupOnRight', except that
the space groups acting from the left are returned.

\>SpaceGroupsByPointGroup( <P> ) O
\>SpaceGroupsByPointGroup( <P>, <norm> ) O
\>SpaceGroupsByPointGroup( <P>, <norm>, <orbsflag> ) O

calls `SpaceGroupByPointGroupOnRight' or `SpaceGroupByPointGroupOnLeft' 
with the same arguments, depending on the value of `CrystGroupDefaultAction'.

\>SpaceGroupTypesByPointGroupOnLeft( <P> ) O
\>SpaceGroupTypesByPointGroupOnLeft( <P>, <orbsflag> ) O

calls either `SpaceGroupTypesByPointGroupOnRight' or
`SpaceGroupTypesByPointGroupOnLeft' with the same arguments, depending
on the variable `CrystGroupDefaultAction'.

%%%%%%%%%%%%%%%%%%%%%%%%%%%%%%%%%%%%%%%%%%%%%%%%%%%%%%%%%%%%%%%%%%%%%%%%%
\Section{Wyckoff positions}

A Wyckoff position of a space group <S> is an equivalence class of
points in Euclidean space, having stabilizers which are conjugate
subgroups of <S>.  Apart from a subset of lower dimension, which
contains points with even bigger stabilizers, a Wyckoff position
consists of an <S>-orbit of some affine subspace <A>. In {\Cryst},
a Wyckoff position <W> is specified by such a representative affine
subspace.

\>WyckoffPositions( <S> ) A

returns the list of Wyckoff positions of the space group <S>.

\beginexample
gap> S := SpaceGroupIT(2,14);
SpaceGroupOnRightIT(2,14,'1')
gap> W := WyckoffPositions(S);
[ < Wyckoff position, point group 3, translation := [ 0, 0 ], 
    basis := [  ] >
    , < Wyckoff position, point group 3, translation := [ 2/3, 1/3 ], 
    basis := [  ] >
    , < Wyckoff position, point group 3, translation := [ 1/3, 2/3 ], 
    basis := [  ] >
    , < Wyckoff position, point group 2, translation := [ 0, 0 ], 
    basis := [ [ 2, 1 ] ] >
    , < Wyckoff position, point group 1, translation := [ 0, 0 ], 
    basis := [ [ 1, 0 ], [ 0, 1 ] ] >
     ]
\endexample

In the previous example, <S> has three kinds of special points
(the basis is empty), whose representatives all have a stabilizer 
with the same point group (with label 1), one kind of special line
(the basis has length 1), and the general position.

\>WyckoffPositionsByStabilizer( <S>, <sub> ) O

where <S> is a space group and <sub> a subgroup of the point group or
a list of such subgroups, determines only the Wyckoff positions whose
representatives have a stabilizer with a point group equal to the 
subgroup <sub> or contained in the list <sub>, respectively.

\beginexample
gap> sub := Group([ [ [ 0, -1 ], [ -1, 0 ] ] ]);
Group([ [ [ 0, -1 ], [ -1, 0 ] ] ])
gap> IsSubgroup( PointGroup( S ), sub );
true
gap> WyckoffPositionsByStabilizer( S, sub );
[ < Wyckoff position, point group 1, translation := [ 0, 0 ], 
    basis := [ [ 1, -1 ] ] >
     ]
\endexample

\>IsWyckoffPosition( <obj> ) R

checks whether <obj> is a Wyckoff position.

\beginexample
gap> ForAll( W, IsWyckoffPosition );
true
\endexample

\>WyckoffBasis( <W> ) O

returns a basis of the representative affine subspace of the Wyckoff 
position <W>.

\beginexample
gap> WyckoffBasis( W[4] );
[ [ 2, 1 ] ]
\endexample

\>WyckoffTranslation( <W> ) O

returns a point of the representative affine subspace of the Wyckoff 
position <W>.

\beginexample
gap> WyckoffTranslation( W[3] );
[ 1/3, 2/3 ]
\endexample

\>WyckoffSpaceGroup( <W> ) O

returns the space group of which <W> is a Wyckoff position.

\beginexample
gap> WyckoffSpaceGroup( W[1] );
SpaceGroupOnRightIT(2,14,'1')
\endexample

\>WyckoffStabilizer( <W> ) O

returns the stabilizer of the (generic) points in the representative
affine subspace of the Wyckoff position <W>. This stabilizer is a
subgroup of the space group of <W>, and thus an `AffineCrystGroup'.

\beginexample
gap> stab := WyckoffStabilizer( W[4] );
Group([ [ [ 1, 1, 0 ], [ 0, -1, 0 ], [ 0, 0, 1 ] ] ])
gap> IsAffineCrystGroupOnRight( stab );
true
\endexample

\>WyckoffOrbit( <W> ) O

determines the orbit of the representative affine subspace <A> of the 
Wyckoff position <W> under the space group <S> of <W> (modulo lattice 
translations). The affine subspaces in this orbit are then converted 
into a list of Wyckoff positions, which is returned. The Wyckoff 
positions in this list are just different representations of <W>. 
Their `WyckoffBasis' and `WyckoffTranslation' are chosen such that 
the induced parametrizations of their representative subspaces are 
mapped onto each other under the space group operation.

\beginexample
gap> orb := WyckoffOrbit( W[4] );
[ < Wyckoff position, point group 2, translation := [ 0, 0 ], 
    basis := [ [ 2, 1 ] ] >
    , < Wyckoff position, point group 2, translation := [ 0, 0 ], 
    basis := [ [ -1, 1 ] ] >
    , < Wyckoff position, point group 2, translation := [ 0, 0 ], 
    basis := [ [ -1, -2 ] ] >
     ]
gap> Set(orb);
[ < Wyckoff position, point group 2, translation := [ 0, 0 ], 
    basis := [ [ 2, 1 ] ] >
     ]
\endexample

\>WyckoffGraph( <W> [, def ] ) O
\>WyckoffGraph( <S> [, def ] ) O

displays the incidence relations of a set of Wyckoff positions graphically.
This function is available only under {\XGAP}. In the first form, <W> is
a list of Wyckoff positions, which must belong to the same space group.
In the second form, <S> is a space group; in this case, the function is
applied to the complete list of Wyckoff positions of <S>. In both forms,
a second argument, <def>, is possible, which is a record with optional
components `title', `width' and `height', specifying the title, width
and height of the graphic sheet on which the graph will be displayed.

Each vertex of the graph represents a Wyckoff position. Vertices are
arranged in horizontal layers, determined by the dimension <s> of the 
Wyckoff position and the size <s> of its stabilizer. For each layer, 
the list <[ d, s ]> is displayed at the right border of the graphic 
sheet. The vertical positions of the layers are ordered according to 
the dimension of the Wyckoff position (primary criterion, smaller 
dimension above) and the size of the stabilizer (secondary criterion, 
bigger stabilizer above). Two Wyckoff positions are connected if the 
closure of the lower one contains the upper one. Two Wyckoff positions
are connected by a line only if there is no Wyckoff position in between. 
The connection line is labelled with the number of affine subspaces 
contained in the lower Wyckoff position that contain a fixed 
representative affine subspace of the upper Wyckoff position.
For instance, if the lower Wyckoff position consists of a space
group orbit of lines (and thus the upper one of an orbit of points), 
the label of the connection line is the number of lines in the orbit 
which cross a fixed representative point of the upper Wyckoff position.

The initial layout of the graph is not always optimal. In particular,
several connection lines can be drawn on top of each other, so that
it is not easy to see who is connected with whom. With the left
mouse button, the graph can be rearranged, however. Just drag each 
vertex to a more suitable place. Note, however, that a vertex can not 
leave its layer. For more details, please consult the {\XGAP} manual.

By right-clicking on a vertex, a popup menu with information on the
Wyckoff position of that vertex appears. It informs on the size of
the `WyckoffStabilizer', the dimension of the Wyckoff position, 
the length of the `WyckoffOrbit' (modulo lattice translations),
the translation and basis of a representative affine subspace, the
isomorphims type of the `WyckoffStabilizer', and the ConjugacyClassInfo
of the point group <P> of the `WyckoffStabilizer'. The ConjugacyClassInfo
lists for each conjugacy class of elements of <P> the number of that 
class, the order, trace and determinant of its elements, and the 
size of the class. This information is useful to identify the 
geometric operation of the stabilizer. The isomorphism type and
ConjugacyClassInfo may not be displayed initially. It this case,
they can be obtained by left-clicking on them, or by left-clicking
on the button labelled <all>. Unfortunately, the popup window 
cannot be resized automatically, and since the ConjugacyClassInfo 
needs several lines for the display, the information may be hidden 
behind the border of the window. You will have to use the slider of
the popup window to make it visible, or resize the window with the
help of your window manager. Alternatively, you can right-click 
again on the same vertex, in which case a new popup window of 
sufficient size appears. 


%%%%%%%%%%%%%%%%%%%%%%%%%%%%%%%%%%%%%%%%%%%%%%%%%%%%%%%%%%%%%%%%%%%%%%%%%
\Section{Normalizers}

\atindex{normalizer!of an AffineCrystGroup}%
{@normalizer!of an \noexpand`AffineCrystGroup'}

At present, most of the functions in this section require that the 
{\GAP} package {\CARAT} is installed (and compiled). Otherwise, they 
are available only for space groups from the crystallographic 
groups catalogue or the International Tables (section "International
Tables").

\>NormalizerPointGroupInGLnZ( <P> ) A

returns the normalizer of the `PointGroup' <P> in the group of 
all unimodular transformations of the lattice spanned by the
`InternalBasis' <B> of the `AffineCrystGroup' <S> of <P>. 
If <S> is in standard representation, this is the same as 
`Normalizer( GL(dim,Integers), P )', otherwise it is
`Normalizer( GL(dim,Integers), P^(B^-1) )^B'. This notion
probably makes sense only if <S> is a space group. Note that 
<P> must have elements with integer entries (which is the case 
if <S> is a space group).

\>CentralizerPointGroupInGLnZ( <P> ) A

returns the centralizer of the `PointGroup' <P> in the group of 
all unimodular transformations of the lattice spanned by the
`InternalBasis' <B> of the `AffineCrystGroup' <S> of <P>. 
If <S> is in standard representation, this is the same as 
`Centralizer( GL(dim,Integers), P )', otherwise it is
`Centralizer( GL(dim,Integers), P^(B^-1) )^B'. This notion
probably makes sense only if <S> is a space group. Note that 
<P> must have elements with integer entries (which is the case 
if <S> is a space group).

\index{normalizer!in translation group}
\>TranslationNormalizer( <S> ) F

returns the normalizer of the space group <S> in the full translation
group. At present, this function is implemented only for space groups,
not for general `AffineCrystGroups'.  The translation normalizer <TN>
of <S> may contain a continuous subgroup <C>.  A basis of the space of
such continuous translations is bound in `TN!.continuousTranslations'.
Since this subgroup is not finitely generated, it is *not* contained
in the group generated by `GeneratorsOfGroup( <TN> )'. Properly speaking, 
the translation normalizer is the span of <TN> and <C> together.

\index{normalizer!in affine group}
\>AffineNormalizer( <S> ) F

returns the affine normalizer of the space group <S>.  The affine
normalizer <AF> contains the translation normalizer as a subgroup.
Similarly as with `TranslationNormalizer', the subgroup <C> of
continuous translations, which is not finitely generated, is not part
of the group that is returned.  However, a basis of the space of
continuous translations is bound in the component 
`AF!.continuousTranslations'.

\>AffineInequivalentSubgroups( <S>, <sub> ) F

takes as input a space group <S> and list of subgroups of <S>,
and returns a sublist of affine inequivalent subgroups.
Note that the affine normalizer of <S> must be discrete in the
current implementation. If it is not, `fail' is returned. 

For two space groups <S1> and <S2> of the same dimension (and
acting from the same side), 

\>ConjugatorSpaceGroups( <S1>, <S2> ) F

returns an affine matrix <m> such that `S1\^{}m = S2', of `fail'
if no such matrix exists, i.e., if the two space groups are not
equivalent. This function requires that the {\GAP} package {\CARAT} is
installed (and compiled).

%%%%%%%%%%%%%%%%%%%%%%%%%%%%%%%%%%%%%%%%%%%%%%%%%%%%%%%%%%%%%%%%%%%%%%%%%
\Section{Color groups}

A color group <C> is a group whose elements are colored in the following 
way. The elements having the same color as the identity element `One(C)' 
form a subgroup <H> of finite index <n>. <H> is called the `ColorSubgroup' 
of <C>. Elements of <C> have the same color if and only if they are 
in the same right coset of <H> in <C>. The labelling of the colors,
which runs from 1 to <n>, is determined by a fixed labelling of the 
right cosets of <H>. The list of right cosets of <H> is stored in
the attribute `ColorCosetList'. The color of the elements of a 
coset corresponds to the position of the coset in that list. Elements 
of <H> by definition have color 1, i.e., the coset with representative 
`One(C)' is always the first element of the `ColorCosetList' of <C>. 
Color groups which have a parent inherit their coloring from that parent, 
including the labelling of the colors. As with other groups, color groups 
having no parent are their own parent.

Right multiplication by a fixed element <g> of <C> induces a permutation 
<p(g)> of the colors of the parent of <C>.  This defines a natural 
homomorphism of <C> into the symmetric group of degree <n>. The image 
of this homomorphism is called the `ColorPermGroup' of <C>, and the 
homomorphism to it is called the `ColorHomomorphism' of <C>.

\>ColorGroup( <G>, <H> ) F

constructs a colored copy of <G>, with color subgroup <H> (which
should have finite index in <G>). Color groups constructed in this way
are always their own parent. It is not possible to set their
parent attribute to a different value.

Groups which may be colored include, in particular, `AffineCrystGroups',  
but coloring of any finite group should work as well.

\>IsColorGroup( <G> ) P

checks whether <G> is a color group.

\>ColorSubgroup( <G> ) A

returns the color subgroup of <G>.

\>ColorCosetList( <G> ) A

returns the color labelling cosets of <G>.

\>ColorOfElement( <G>, <elem> ) F

returns the color of an element of <G>.

\>ColorPermGroup( <G> ) A

returns the ColorPermGroup of <G>, which is the permutation group 
induced by <G> acting on the colors of the parent of <G>.

\>ColorHomomorphism( <G> ) A

returns the homomomorphism from <G> to its `ColorPermGroup'.

\>Subgroup( <C>, <elems> )!{for color groups} O

where <C> is a color group, returns the colored subgroup <U> of <C>
generated by <elems>. The parent of <U> is set to the parent of <C>, 
from which the coloring of <U> is inherited.

\beginexample
gap> G := Group(  (1,2,3), (2,3,4) );
Group([ (1,2,3), (2,3,4) ])
gap> H := Group( (1,2,3) ); 
Group([ (1,2,3) ])
gap> C := ColorGroup( G, H );
Group([ (1,2,3), (2,3,4) ])
gap> ColorSubgroup( C ) = H;
true
gap> ColorCosetList( C );
[ RightCoset(Group( [ (1,2,3) ] ),()), RightCoset(Group( [ (1,2,3) ] ),(1,2)
    (3,4)), RightCoset(Group( [ (1,2,3) ] ),(1,3)(2,4)), 
  RightCoset(Group( [ (1,2,3) ] ),(1,4)(2,3)) ]
gap> List( last, x -> ColorOfElement( C, Representative(x) ) );
[ 1, 2, 3, 4 ]
gap> U := Subgroup( C, [(1,3)(2,4)] );
Group([ (1,3)(2,4) ])
gap> IsColorGroup( U );
true
gap> ColorSubgroup( U );
Group(())
gap> ColorCosetList( U );
[ RightCoset(Group( () ),()), RightCoset(Group( () ),(1,3)(2,4)) ]
gap> List( last, x -> ColorOfElement( U, Representative(x) ) );
[ 1, 3 ]
\endexample

%%%%%%%%%%%%%%%%%%%%%%%%%%%%%%%%%%%%%%%%%%%%%%%%%%%%%%%%%%%%%%%%%%%%%%%%%
\Section{Colored AffineCrystGroups}

If <C> is a colored `AffineCrystGroup' whose `ColorSubgroup' is 
lattice-equal (translationengleich) with <C>, the `PointGroup' of 
<C> can consistently be colored. In that case,

\>PointGroup( C )!{for a colored `AffineCrystGroup'} A

returns a colored point group. Otherwise, the `PointGroup' of <C> is
an ordinary, uncolored group.

\beginexample
gap> S := SpaceGroupIT( 2, 10 );                                  
SpaceGroupOnRightIT(2,10,'1')
gap> m := MaximalSubgroupClassReps( S, rec( primes := [2] ) );    
[ <matrix group with 4 generators>, <matrix group with 3 generators>, 
  <matrix group with 4 generators> ]
gap> List( last, x -> TranslationBasis(x) = TranslationBasis(S) );
[ false, true, false ]
gap> C := ColorGroup( S, m[1] );; IsColorGroup( PointGroup( C ) );
false
gap> C := ColorGroup( S, m[2] );; IsColorGroup( PointGroup( C ) );
true
\endexample

Two colorings of a *space group* <S> are *equivalent* if the two
`ColorSubgroups' are conjugate in the affine normalizer of <S>.
For instance, a list of inequivalent index-2 `ColorSubgroups' of 
<S> can be obtained with the following code:

\index{colorings!inequivalent!for space group}
\beginexample
gap> sub := MaximalSubgroupClassReps( S, rec( primes := [2] ) );
[ <matrix group with 4 generators>, <matrix group with 3 generators>, 
  <matrix group with 4 generators> ]
gap> sub := Filtered( sub, s -> IndexInParent( s ) = 2 );
[ <matrix group with 4 generators>, <matrix group with 3 generators>, 
  <matrix group with 4 generators> ]
gap> sub := AffineInequivalentSubgroups( S, sub );
[ <matrix group of size infinity with 4 generators>, 
  <matrix group of size infinity with 3 generators> ]
\endexample

Note that `AffineInequivalentSubgroups' requires the {\GAP} package 
{\CARAT} to be installed. Otherwise, this function is supported only 
for `AffineCrystGroups' constructed from the crystallographic groups 
catalog.

%%%%%%%%%%%%%%%%%%%%%%%%%%%%%%%%%%%%%%%%%%%%%%%%%%%%%%%%%%%%%%%%%%%%%%%%%
\Section{International Tables}

For the user's convenience, a table with the 17 plane groups and the
230 space groups is included in {\Cryst}. These groups are given in 
exactly the same settings (i.e., choices of basis and origin) as in 
the International Tables. Space groups with a centered lattice are
therefore given in the non-primitive basis crystallographers are
used to. This is in contrast to the crystallographic groups catalogue,
where always a primitive basis is used.

For some of the 3D space groups, two different settings are available.
The possible settings are labelled with the characters `\pif{1}\pif',
`\pif{2}\pif',`\pif{b}\pif',`\pif{c}\pif',`\pif{h}\pif' and `\pif{r}\pif'.
If only one setting is available, it is labelled `\pif{1}\pif'. For
some space groups there exists a point with higher symmetry than the
origin of the `\pif{1}\pif' setting. In such cases, a second setting
`\pif{2}\pif' is available, which has this high symmetry point as
origin. This second setting `\pif{2}\pif' then is the default setting.
Space groups which have a unique axis can have this axis in <b>
direction (setting`\pif{b}\pif') or <c> direction (setting
`\pif{c}\pif'). `\pif{b}\pif' is the default setting. Rhombohedral 
space groups are given in a hexagonal basis (setting `\pif{h}\pif') and
in a rhombohedral basis (setting `\pif{r}\pif'). `\pif{h}\pif' is the 
default setting.

\>SpaceGroupSettingsIT( <dim>, <nr> ) F

returns a string, whose characters label the available settings of
the space group with number <nr> and dimension <dim>.

\> SpaceGroupOnRightIT( <dim>, <nr> ) F
\> SpaceGroupOnRightIT( <dim>, <nr>, <setting> ) F

returns space group number <nr> in dimension <dim> in the
representation acting on the right. In the third argument,
the desired setting can be specified. Otherwise, the space 
group is returned in the default setting for that space group.

\> SpaceGroupOnLeftIT( <dim>, <nr> ) F
\> SpaceGroupOnLeftIT( <dim>, <nr>, <setting> ) F

returns space group number <nr> in dimension <dim> in the
representation acting on the left. In the third argument,
the desired setting can be specified. Otherwise, the space 
group is returned in the default setting for that space group.

\> SpaceGroupIT( <dim>, <nr> ) F
\> SpaceGroupIT( <dim>, <nr>, <setting> ) F

returns either `SpaceGroupOnRightIT' or `SpaceGroupOnLeftIT' with
the same arguments, depending on the value of `CrystGroupDefaultAction'.

\beginexample
gap> SpaceGroupSettingsIT( 3, 146 );
"hr"
gap> SpaceGroupOnRightIT( 3, 146 );        
SpaceGroupOnRightIT(3,146,'h')
gap> SpaceGroupOnRightIT( 3, 146, 'r' );
SpaceGroupOnRightIT(3,146,'r')
\endexample



